\documentclass{article}
\usepackage[utf8]{inputenc}

\title{Relazione progetto calcolo numerico}
\author{Koci Erik | Girolimetto Mattia | Andreea Scrob }
\date{January 2022}

\begin{document}

\maketitle

\section{Intestazione}
\textbf{Nome del progetto:} Deblur Immagini\\
\textbf{cognome e nome autori:}
\begin{itemize}
    \item Koci Erik, 0000997662
    \item
    \item
\end{itemize}

\section{Problema}
\begin{enumerate}
    \item Riportare e commentare i risultati ottenuti nei punti 2. 3. (e 4.) su un immagine del set creato e sualtre due immagini in bianco e nero (fotografiche/mediche/astronomiche ecc.)
    \item Riportare delle tabelle con le misure di PSNR e MSE ottenute al variare dei parametri (dimensionekernel, valore di sigma, la deviazione standard del rumore, il parametro di regolarizzazione).
    \item Calcolare sull’intero set di immagini medie e deviazione standard delle metriche per alcuni valori fissatidei parametri.
    \item Analizzare su 2 esecuzioni le propriet`a dei metodi numerici utilizzati (gradiente coniugato e gradiente)in termini di numero di iterazioni, andamento dell'errore, della funzione obiettivo, norma del gradiente.3
\end{enumerate}

\end{document}

